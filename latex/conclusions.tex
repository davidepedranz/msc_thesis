\chapter{Conclusions}
\label{chapter:conclusions}
In this work, we implemented an event driven simulator of the low-level Bitcoin protocol using the PeerSim framework.
We used the simulator to run Bitcoin networks of different sizes under various network conditions: at rest, with generalized delays and under attack.
In particular, we focused on the Balance attack, which consist in partitioning the nodes into \num{2} groups by delaying communications between peers in different partitions.
Finally, we implemented some variants of the original attack and evaluated their performances.

Our experiments show the Bitcoin protocol is robust under normal network conditions and in presence of random message losses, thanks to the gossip-style underlying protocol.
However, the network produces a significant number of forks in presence of diffuse delays.
The situation is even worse under a Balance attack:
delays on blocks broadcast between nodes of different partitions higher the probability of forks in the network.

The variants of the Balance attack we tested do not improve the attack's effect.
Randomly message dropping is totally ineffective, unless the attacker drops at least \num{80}\% to \num{90}\% of all messages exchanged between nodes in different partitions:
if such an attack was feasible, the attacker could simply drop all messages, isolating completely the \num{2} groups from each other.

Higher number of partitions seems to have the same effect as the base attack:
the forks distribution for a number of partitions up to \num{10} does not differ significantly from the one with just \num{2} of them.
An attacker might consider to create more than \num{2} groups of nodes if this were easier in a particular context.

Balance attacks are very effective, but difficult to achieve in practice, since an attacker needs to control a significant number of the links that connect Bitcoin nodes over the Internet.
Still, similar attacks have been performed in the past against pools of miners:
weaknesses in the \ac{BGP} allowed an attacker to hijack connections between miners and pool coordinators, taking control of the victims' mining power.
Since all communications in Bitcoin are not encrypted and use a simple TCP connection, the same idea can be easily used by a powerful attacker to mount a Balance attack.


\section{Future work}
As stated in the original paper \cite{balance_attack_2017}, all \textit{forkable} blockchains are potentially vulnerable to a Balance attack.
Some new cryptocurrencies try to solve most problems of Bitcoin and similar digital currencies by almost completely eliminating the possibility of forks.
For example, Algorand \cite{algorand_2017} defines a new byzantine consensus algorithm based a novel cryptographic technique called Verifiable Random Functions to build a fast protocol that can confirm transactions with a small latency and avoid the generation of forks even under attack.
Possible future works include extending our simulator to the Algorand protocol and experimentally verifying if the claimed properties yield in practice.
