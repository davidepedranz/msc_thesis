\chapter{Attacks}
\label{chapter:attacks}
Various attacks has been performed against Bitcoin in the past.
Attacks can be performed at different levels with very different approaches and can have a wide range of effects on Bitcoin users.
This chapter starts from an overview of the most relevant ones.
At the end, it describes in details the Balance attack, whose analysis is the main object of the simulations.

\section{Double Spending}
\label{sec:double-spending}
Double-spending is the result of successfully spending some money more than once.
The attack is the oldest known one against Bitcoin (it is even partially discussed in the original Bitcoin paper \cite{bitcoin_2009}).
The attack has been analyzed many times in the literature \cite{double_spending_fast_payments, double_spending_two_for_one, double_spending_bitcoin_economics, double_spending_fast_analysis_2014} and has been reported to be quite easy to mount for fast payments \cite{double_spending_fast_payments}.
Real episodes of the attack has been reported multiple times in the past \cite{double_spending_ghash, double_spending_stackexchange}.
Double-spending breaks the most important guarantee that Bitcoin tries to give:
transactions are irreversible and bitcoins can be spent only once.

The attack works by creating \num{2} conflicting transactions (that spend all the bitcoins in the same address) and submitting them to different parties, for example \texttt{tx-1} to a merchant and \texttt{tx-2} to the Bitcoin network:
if the merchant accepts \texttt{tx-1} but the Bitcoin network receives \texttt{tx-2} before \texttt{tx-1}, it is likely that \texttt{tx-2} will be stored in the blockchain, while \texttt{tx-1} will be discarded, since not compatible with \texttt{tx-2}.
The attacker only pretends to spend some bitcoins to pay the merchant, but it actually does not spend anything:
it gets the purchase for free, while the merchant never receives the due bitcoins.
The double-spending attack can be performed directly in some cases, or it can be the result of more complex attacks.
There are many variants of double-spending, we analyze the most common ones here.

\subsection{Race Attack}
\label{sub:race-attack}
The race attack involves merchants that immediately accepts payments, without waiting for the unconfirmed transaction to be securely store in some block in the blockchain \cite{bitcoin_wiki_irreversible_transactions}.
The race attack has a high degree of success for an attacker \cite{double_spending_two_for_one}.
The attack is illustrated in \cref{fig:race-attack} and works as follows:
\begin{itemize}
	\item the attacker sends a transaction \texttt{tx-1} to the merchants;
	\item at the same time, the attacker sends a conflicting transaction \texttt{tx-2} to some miner; the conflicting transaction spend the bitcoins stored in the same address used in \texttt{tx-1} and sends them to new address in control of the attacker;
	\item the merchants sees the unconfirmed transaction and accepts the payment; this scenario is more likely for in-shop purchases, where, in many cases, the merchant can not wait for the transaction to be stored in the blockchain;
	\item \texttt{tx-2} is received before \texttt{tx-1} from the miners, so it is more likely to be stored in some block before \texttt{tx-1};
	\item \texttt{tx-2} is stored in a block, while \texttt{tx-1} is rejected since in conflict with \texttt{tx-1}.
	      % \item the attacker receives its bitcoins back, while the merchant does not get anything.
\end{itemize}

\begin{figure}[t]
	\centering
	\vspace*{0.25cm}
	\includegraphics[scale=0.75]{figures/race_attack}
	\vspace*{0.25cm}
	\caption[Illustration of the race attack, a variant of double-spending]{
		Illustration of the race attack, a variant of double-spending.
		The attacker (in red) owns some bitcoins in the address \texttt{0x11}.
		It submits a transaction \texttt{tx-1} to move the money from its address \texttt{0x11} to the merchant's one \texttt{0x99} (in blue).
		At the same time, it submit to the Bitcoin network a conflicting transaction \texttt{tx-2} to move the the money from \texttt{0x11} to a new address \texttt{0x12} in its control.
		If the network includes \texttt{tx-2} in some block before \texttt{tx-1}, \texttt{tx-1} is rejected, and the money stay control of the attacker.
		The color of the arrows indicates the real flow of bitcoins, after that \texttt{tx-2} is stored in the blockchain and \texttt{tx-1} is rejected.
	}
	\label{fig:race-attack}
\end{figure}

A trivial defense for the merchant is to wait for the transaction \texttt{tx-1} to be stored in the blockchain:
if it is rejected because of any conflicts, the merchant can simply refuse the payment and do not sell the good.
Unfortunately, Bitcoin requires on average \SI{10}{minutes} to confirm a transaction, so this technique can not be always applied.
Bitcoin's developers recommendation is also to wait for \num{6} confirmations, i.e. to wait for the block that stored the transaction to be in the longest chain and to have \num{6} following blocks \cite{confirmation}.
Some blocks can be conflicting with each other (if they have the same parent):
only one of them can belong to the longest chain, while the others are said to be ``orphaned'' \cite{orphaned_block} and simply ignored, as illustrated in \cref{fig:orphaned-block}.
Thus, a transaction can be rejected even if it manages to get part of a block, if that block is orphaned.
Waiting for \num{6} confirmations gives a good tradeoff between the time to wait (on average \SI{1}{hour}) and statistical guarantee that the transaction is unlikely to be rejected later \cite{bitcoin_2009}.

\begin{figure}[ht]
	\centering
	\vspace*{0.25cm}
	\includegraphics[scale=1.1]{figures/orphaned_block}
	\vspace*{0.25cm}
	\caption[Illustration of an orphaned block]{
		Illustration of an orphaned block.
		Blocks \texttt{9a} and \texttt{9b} have the same parent.
		Since \texttt{9a} has a child \texttt{10a}, it is on the longest chain.
		Transactions \texttt{tx-6}, \texttt{tx-7} and \texttt{tx-8} from the orphaned block \texttt{9b} are stored in \texttt{10a};
		\texttt{tx-1} (in red) is in conflict with \texttt{tx-2} and is thus rejected.
	}
	\label{fig:orphaned-block}
\end{figure}

\subsection{Finney Attack}
The Finney attack is another variant of double-spending targeting merchants that accepts unconfirmed payments.
It can be performed by an attacker that is able to mine some blocks.
The attack works as follows:
\begin{itemize}
	\item the attacker controls addresses \texttt{addr-a} and \texttt{addr-b};
	\item the attacker mines a new block on the longest chain; the block includes a transactions \texttt{tx-2} that moves all bitcoins from address \texttt{addr-a} to \texttt{addr-b};
	\item before broadcasting the block to the Bitcoin network, the attacker sends a transaction \texttt{tx-1} to a merchant, spending again the bitcoins in \texttt{addr-a};
	\item as soon as the merchants accepts the payment, the attacker broadcasts its block to the network;
	\item \texttt{tx-2} is stored in the blockchain before \texttt{tx-1}, so \texttt{tx-1} is rejected because in conflict with \texttt{tx-2};
	\item the attacker receives its bitcoins back, while the merchant does not get anything.
\end{itemize}
Similarly to the Race attack, the Finney attack relies on conflicting transactions and the significant time required to confirm a transaction with a high confidence in Bitcoin.
The defense for the merchant is the same discussed in \cref{sub:race-attack}.

\section{Majority Attack}
\label{sec:majority-attack}
The majority attack, also known as \num{51}\% attack, refers to a situation in which one miner (or one group of miners) control more that \num{50}\% of the entire network's mining hash rate \cite{majority_investopedia, majority_bitcoin_wiki}.
An attacker with such a computational power can generate blocks faster than the rest of the network, so it can take control of the entire blockchain and each block in the longest chain.
The attacker strategy is to mine its private chain, ignoring the current longest chain and each block honest miner published.
This strategy gives statically guarantees to eventually generate the longest chain, no matter the advantage of honest miners (in terms of the difference in the number of blocks on the current longest chain and the number of blocks in the attacker's chain).
Such an attacker could, potentially, start its own chain from the genesis block and revert the entire history of Bitcoin, as illustrated in \cref{fig:majority-attack}.

There is no defense against this attack:
Bitcoin's security relies on an honest majority of nodes that collectively control more computational power than any attacker (or group of attackers) \cite{bitcoin_2009}.
To be more precise, the system is not vulnerable to the majority attack as long as no single coalition of miners controlling more than \num{50}\% of the total hash rate \cite{bitcoin_wiki_irreversible_transactions}.

\begin{figure}[t!]
	\begin{subfigure}{\textwidth}
		\centering
		\vspace*{0.25cm}
		\includegraphics[scale=0.9]{figures/majority_attack_1}
		\vspace*{0.25cm}
		\caption{
			The attacker starts to mine block (in red) from the genesis (number \texttt{0}, in blue) instead of following the longest chain (in green).
			Honest miners keep following the green chain that finishes with block \texttt{3a}.
		}
		\vspace*{0.75cm}
	\end{subfigure}
	\begin{subfigure}{\textwidth}
		\centering
		\vspace*{0.25cm}
		\includegraphics[scale=0.9]{figures/majority_attack_2}
		\vspace*{0.25cm}
		\caption{
			Eventually, the attacker's chain gets longer than the previous longest chain.
			The attacker broadcasts all blocks in his chain to the Bitcoin network.
		}
		\vspace*{0.75cm}
	\end{subfigure}
	\begin{subfigure}{\textwidth}
		\centering
		\vspace*{0.25cm}
		\includegraphics[scale=0.9]{figures/majority_attack_3}
		\vspace*{0.25cm}
		\caption{
			Honest miners start following the attacker's chain (now in green), since it is not longer than the chain terminating with block \texttt{3a}.
			The oldest chain (now in yellow) is orphaned: all transactions stored in blocks \texttt{1a}, \texttt{2a} and \texttt{3a} are not valid anymore.
			Potentially, the attacker may include transactions conflict with the canceled once in any of the new blocks \texttt{1b}, \texttt{2b}, \texttt{3b} and \texttt{4b}, effectively double-spending some or all of its bitcoins.
		}
		\vspace*{0.25cm}
	\end{subfigure}
	\caption[Illustration of the different phases of a majority attack]{
		Illustration of the different phases of a majority attack, where an attacker takes control of the entire blockchain, starting from the genesis block.
		The attacker is able to revert the history of all transactions stored in the blockchain.
	}
	\label{fig:majority-attack}
\end{figure}

The majority attack allows an attacker to perform the following actions \cite{weaknesses_bitcoin_wiki}:
\begin{itemize}
	\item rewrite the history of the blockchain;
	\item prevent some or all transactions from becoming part of the longest chain;
	\item prevent some or all other miners from attaching new blocks to the longest chain (and thus getting any reward);
	\item reverse transactions it has made in the past, effectively double-spending its bitcoins;
	\item gain the revenue of all new blocks.
\end{itemize}

A majority attack can completely destroy the working of Bitcoin and violate some of its main guarantees (e.g. spending the same bitcoins only once).
Such an attack is thought to be very unlikely \cite{ghash_never_51_attack_cex, ghash_never_51_attack_coindesk}, because of its huge cost in terms of computing power and the risk for an attacker not to gain anything from it:
even if it is theoretically possible to revert all blocks following the genesis, it is incredible expensive.
The attack would immediately destroy the credibility of Bitcoin and has \num{2} possible effects:
\begin{enumerate}
	\item bitcoins value drops completely, and the attacker had no real reward;
	\item the rules of the software are changed by the community to revert the attack \cite{weaknesses_bitcoin_wiki}.
\end{enumerate}

% TODO: this makes formatting better...
\pagebreak

In January 2014, the mining pool \texttt{GHash.IO} reached \num{42}\% of the total Bitcoin hash rate \cite{ghash_fears_51_attack, security_survey_2017}, and later in July 2014 it exceeded the threshold of \num{51}\% \cite{wikipedia_ghash, majority_investopedia, bitcoin_wiki_irreversible_transactions}.
Even though there is no evidence of a majority attack to ever been mounted against Bitcoin, the episode was quite controversial in 2014:
a number of miners voluntarily dropped out of the pool and \texttt{GHash.IO} implemented a mechanism to prevent the situation to ever happen again in the future \cite{ghash_51_percent_extremetech, ghash_commits_to_40_percent_coindesk, ghash_commits_to_40_percent_arstechnica}.

\section{Selfish Mining}
\label{sub:selfish}
Selfish Mining \cite{selfish_mining_acm} is a strategy where a group of miners strategically chooses when to submit the new blocks to the public chain, rather than submitting them immediately upon discovery.
The selfish miners keep the discovered blocks private, intentionally forking the blockchain and mining on their own private branch.
The honest nodes do not know about the blocks discovered by the selfish miners, so they continue to mine on the public chain.
If the public longest chain approaches the selfish miners' private branch length, they reveal some blocks to surpass the public chain again.
The selfish miners causes the honest miners to waste a lot of computational power on obsolete chains, since the longest one is not yet public.
The idea is illustrated in \cref{fig:selfish-mining}.

\begin{figure}[h!]
	\begin{subfigure}{\textwidth}
		\centering
		\vspace*{0.25cm}
		\includegraphics[scale=0.9]{figures/selfish_1}
		\vspace*{0.25cm}
		\caption{
			The selfish miners secretly mine block \texttt{13b} from the current public longest chain (in green).
			The honest miners do not know about block \texttt{13b}, so they keep working from block \texttt{12}.
		}
		\vspace*{0.75cm}
	\end{subfigure}
	\begin{subfigure}{\textwidth}
		\centering
		\vspace*{0.25cm}
		\includegraphics[scale=0.9]{figures/selfish_2}
		\vspace*{0.25cm}
		\caption{The selfish miners manage to secretly mine another block \texttt{14b} and keep it secret (secret blocks are indicated by the red dashed box).}
		\vspace*{0.75cm}
	\end{subfigure}
	\begin{subfigure}{\textwidth}
		\centering
		\vspace*{0.25cm}
		\includegraphics[scale=0.9]{figures/selfish_3}
		\vspace*{0.25cm}
		\caption{
			Honest miners manage to mine block \texttt{13a} and publish it.
			Selfish miners reveal the secret chain, which is longer than the public longest chain.
			Honest miners discard block \texttt{13a} (in yellow) and start to follow the new longest chain (block \texttt{14b}).
			The computational power used to mine block \texttt{13a} gets wasted and the honest miners do not get any revenue.
		}
		\vspace*{0.25cm}
	\end{subfigure}
	\caption[Illustration of the Selfish Mining strategy]{Illustration of the Selfish Mining strategy.}
	\label{fig:selfish-mining}
\end{figure}

Selfish Mining has been proposed in 2013 by Ittay Eyal and Emin G\"un Sirer \cite{selfish_mining}.
In their paper, they show that both the selfish and the honest miners waste some resources, but the honest miners waste proportionally more:
overall, the rewards share for the selfish miners is higher than the proportion of computational power they own.
In practice, this gives the selfish miners an advantage over the others and encourages rational miners to join the group of selfish miners.
According to the paper, relatively small group of miners can adopt the selfish strategy and attract many other miners, with the risk of gaining the majority of the mining power (see \cref{sec:majority-attack}).

\bigskip
Some work has been done in academics to investigate alternative mining strategy \cite{other_mining_strategies_2014, block_hiding_strategies_2014}.

Nayak Kartik et at. have proposed Stubborn Mining \cite{stubborn_mining_2016}, a strategy heavily inspired by Selfish Mining, but that accepts some more risk (e.g. keep mining on chains shorter than the longest one) in return to a higher expected revenue.
They claim that stubborn mining strategies can beat Selfish Mining by up to \num{25}\%.

Ayelet Sapirshtein, Yonatan Sompolinsky and Aviv Zohar have investigated the space of possible selfish mining strategies, looking for the optimal strategy for selfish miners \cite{optimal_selfish_mining_2016}.
They also show that selfish mining strategies can be combined with double-spending attempts to make them even more profitable for selfish miners.


\section{Eclipse Attack}
\label{sub:eclipse}
The general idea of the Eclipse Attack \cite{eclipse_overlay_2006} is to force a victim to connect only nodes under the control of the attacker:
if the attacker succeeds in doing that, it is able to control all the network traffic to and from the victim.
In the specific case of Bitcoin, the attacker can mount many different attacks against the victim, for example a double-spending against a merchant.

\begin{figure}[h!]
	\begin{subfigure}{.45\textwidth}
		\vspace*{0.25cm}
		\includegraphics[width=\columnwidth]{figures/eclipse_1}
		\vspace*{0.1cm}
		\caption{
			At the beginning, the victim is connected to some honest miners in the Bitcoin network.
			The attacker opens new connections against the victim from many different malicious nodes under its control.
			If the attack succeeds, the victim will eventually forget the addresses of honest nodes and fill its tables with addresses of malicious nodes.
		}
	\end{subfigure}
	\hfill
	\begin{subfigure}{.45\textwidth}
		\vspace*{0.25cm}
		\includegraphics[width=\columnwidth]{figures/eclipse_2}
		\vspace*{0.1cm}
		\caption{
			When the victim reboots, it connects to known nodes.
			Since its tables are full of addresses of hosts under the control of the attacker, the victim will connect only to malicious nodes.
			The attacker can now decide which messages to forward to the victim and which not to:
			the attacker can effectively eclipse the victim from the peer-to-peer network.
		}
	\end{subfigure}
	\caption[Illustration of an Eclipse Attack]{Illustration of an Eclipse Attack.}
	\label{fig:eclipse}
\end{figure}

\bigskip
Each Bitcoin node maintains \num{2} tables of addresses \cite{eclipse_attack_2015}:
the \texttt{tried} table contains addresses of peers that were successful connected with the current node at least one in the past;
the \texttt{new} table contains addresses discovered through any of the peers-discovery strategies used by Bitcoin (see \cref{sec:discovery}) to whom the node has not yet initiated a successful connection.

A Bitcoin running the original client \cite{bitcoin_github} with the default settings can have up to \num{8} outgoing and up to \num{117} unsolicited incoming connections, for a total of \num{125} \cite{deanonymisation_2014, eclipse_attack_2015} (see \cref{chapter:protocol} for the details).
Connections are long-lived, i.e. they are interrupted only if either one of the \num{2} nodes goes down or in case of networking problems.
Each Bitcoin node always tries to have exactly \num{8} outgoing connections, in order to maintain a solid network topology.
If the number of outgoing connections gets less than \num{8}, the node randomly chooses a new IP address from the \texttt{new} table and tries to connect to the corresponding node:
if the connection is not successful, the node tries with the next address.

\bigskip
An Eclipse Attack against Bitcoin \cite{eclipse_attack_2015} is illustrated in \cref{fig:eclipse} and works as follows:
\begin{itemize}
	\item the attacker rapidly and repeatedly opens new unsolicited connections to the victim from a set of malicious nodes under its control; when a new connection is establishes, the attacker also sends unsolicited \texttt{ADDR} messages containing a list of up to \num{1000} ``trash'' addresses (IPs of host that do not run Bitcoin or addresses under the control of the attacker);
	\item if the attack succeeds, the victim will eventually replace all addresses stored in both the \texttt{tried} and the \texttt{new} tables with the IP addresses of the malicious nodes;
	\item the attacker forces the victim to restart \cite{cve_bloom_filter_2013, bitcoin_common_vulnerabilities} or simply waits for it to restart naturally, for example after a software update or after some failure;
	\item when the victim restarts, it will connect to \num{8} malicious nodes randomly chooses from the \texttt{tried} and the \texttt{new} tables;
	\item the attacker effectively isolates the victim from the honest nodes in the network and can control which messages, block and transactions to relay, delay or drop.
\end{itemize}

The attacker can mount many different attacks against eclipsed nodes.
If the victim is a miner (or a mining pool), the attacker can adopts strategies similar to the Selfish Mining to waste the computing power of the eclipsed nodes:
the attacker can drop \texttt{BLOCK} messages containing blocks discovered by the victim and forward blocks from the network that conflict with the ones found by the victim.
If the victim is a merchant, the attacker can easily mount double-spending attacks.
It can send a transaction \texttt{tx-1} to the merchant, and a conflicting one \texttt{tx-2} to the Bitcoin network:
since the attacker controls all of merchant's connections, the merchant might not be able to broadcast \texttt{tx-1} to the network and the attacker might be able to double-spend its bitcoins with \texttt{tx-2}.


\section{BGP Hijacking}
\ac{BGP} is a standardized network protocol used to exchange routing information between \ac{AS} \cite{autonomous_systems_wikipedia} of different \ac{ISP} and organizations on the Internet \cite{rfc4271, bgp_wikipedia}.
The primary function of \ac{BGP} is to exchanges information about network reachability with other \ac{BGP} systems.
The information collected and distributed by the protocol are used to construct a graph of connectivity between different \ac{AS}, which can be used to make optimal routing decisions among the networks of different parties.
\ac{BGP} routing also depends on policies and rules configured by network administrators.

\bigskip
While not directly related to Bitcoin, \ac{BGP} is fundamental for the working of Internet, and thus for each distributed systems build on top of that.
Under some conditions, attacks against \ac{BGP} can affect Bitcoin and mine some of the guarantees given by the protocol.
Researchers at Dell SecureWorks Counter Threat Unit discovered an attacker that repeatedly hijacked traffic destined to networks belonging to Amazon, Digital Ocean, OVH, and other large hosting companies between February and May 2014 \cite{bgp_hijacking_secureworks}:
they registered \num{51} compromised networks from \num{19} different \ac{ISP}.
Attacks with a similar scale were also reported between October 2015 and April 2016 \cite{hijacking_bitcoin_2017, bgpstream}.

% TODO: this makes formatting better...
\pagebreak

\begin{figure}[h!]
	\begin{subfigure}{\textwidth}
		\centering
		\vspace*{0.25cm}
		\includegraphics[width=\columnwidth]{figures/bgp_1}
		\vspace*{0.25cm}
		\caption{
			Broadcast of the malicious route.
			The legitimate miner (in green) has IP address \texttt{1.1.1.2} on the network \texttt{1.1.1.1/16} of \texttt{AS2}.
			\texttt{AS2} broadcasts its networks to the neighboring \ac{AS}.
			The attacker (in red) tries to hijack the victim's connection:
			since \texttt{AS4} is paired with \texttt{AS3}, the BGP message from the attacker is accepted.
			The malicious route is more specific than the legitimate one, so it is overridden in the \texttt{AS4} routing tables.
		}
		\vspace*{0.25cm}
	\end{subfigure}
	\begin{subfigure}{\textwidth}
		\centering
		\vspace*{0.25cm}
		\includegraphics[width=\columnwidth]{figures/bgp_2}
		\vspace*{0.25cm}
		\caption{
			The victim tries to connect to the legitimate mining pool server at IP \texttt{1.1.1.1}.
			The IP packets are correctly routed to \texttt{AS4}, but then \texttt{AS4} sends them to \texttt{AS3} instead of \texttt{AS2}:
			the victim connects to the attacker instead.
			The routing is indicated with red arrows;
			green arrows with red crosses indicate what the correct routing would be.
		}
		\vspace*{0.25cm}
	\end{subfigure}
	\caption[Illustration of a BPG Hijacking attack against a Bitcoin miner]{Illustration of a BPG Hijacking attack against a Bitcoin miner.}
	\label{fig:bpg-hijacking}
\end{figure}

A \ac{BGP} attack requires a malicious \ac{AS} and some manual configuration.
A network administrator of the malicious \ac{AS} must configure \ac{BGP} to announce wrong ranges of IP addresses:
the other \ac{AS} will learn wrong routes and override the correct routing rules.
If the attack is successful, some Internet traffic destined to the hijacked networks goes through the malicious \ac{AS} and can thus be intercepted, analyzed and manipulated by the attacker in any ways.
Since all the Bitcoin traffic is in clear and there in no authentication between different peers, Bitcoin is vulnerable to hijacking \cite{hijacking_bitcoin_2017}.
Similarly, traffic of the Stratum protocol used in by mining pools is vulnerable to routing attacks.

\bigskip
\cref{fig:bpg-hijacking} shows an example of \ac{BGP} hijacking used against a Bitcoin miner running the Stratum protocol.
The attack works as follows:
\begin{itemize}
	\item the miner continuously connects to a legitimate pool, asking for tasks to perform;
	\item the attacker publishes new \ac{BGP} routes to intercept the victim's Stratum network traffic;
	\item when the miner attempts to connect to a legitimate pool, some \ac{BGP} route directs the traffic to a pool controlled by the attacker;
	\item the malicious pool issues a \texttt{client.reconnect} \cite{stratum_manual} command to instruct the miner to connect to a new pool maintained by the attacker;
	\item the victim connects to the second malicious pool;
	\item the attacker ceases the attack, since \ac{BGP} hijacking is not needed anymore at this stage;
	\item the miner performs the assigned tasks correctly, but does not get any revenue from the malicious pool, effectively wasting its computing power for free.
\end{itemize}

\bigskip
In the paper ``Hijacking Bitcoin: Routing Attacks on Cryptocurrencies'', Maria Apostolaki et at. show that large scale networks attack against Bitcoin are possible and have been already done in the past.
They claim that anyone with access to a BGP-enabled network and able to hijack less than \num{900} prefixes (~\num{0.15}\% of the total) can perform attacks commonly believed to be hard, such as isolating \num{50}\% of the mining power.
Attackers controlling some networks can potentially:
\begin{itemize}
	\item force miners to connect to a malicious pool which exploits their mining power for free;
	\item isolate part of the network by delaying or dropping messages to or from the victims (see the Eclipse Attack described in \cref{sub:eclipse});
	\item perform double-spend attacks against merchants;
	\item take control of a significant amount of computing power and attempt various kind of mining attacks (see \cref{sub:selfish}).
\end{itemize}


\section{Balance Attack}
The Balance attack is a new attack proposed by Christopher Natoli and Vicent Gramoli first in \num{2016} in the technical report \cite{balance_attack_report_2016} and later in \num{2017} in their paper ``The Balance Attack or Why Forkable Blockchains Are Ill-Suited for Consortium'' \cite{balance_attack_2017}.
The attack targets ``forkable'' blockchains, i.e. a blockchain that allow multiple branches at the same time.
The most famous examples of systems using a forkable blockchains are Bitcoin \cite{bitcoin_2009} and Ethereum \cite{ethereum_2014}.

\bigskip
Forkable blockchains implement some strategy to solve conflicts:
for example, Bitcoin always follows the longest chain if multiple branches exists.
The rule guarantees that the system will eventually agree on the same longest chain, and thus on the same set of transactions stored in the blockchain.
Unfortunately, blocks in branches different from the longest chain are ``wasted'', since their content is not recognized by the network.
Wasted blocks are probably the biggest problem of forkable blockchains:
they waste computational power of the network and facilitate double-spending attacks \cite{double_spending_fast_analysis_2014} (see \cref{sec:double-spending}).

Except for attacks, network delays are the main cause of forks in the blockchain \cite{information_propagation_2013}.
Honest miners immediately stop the mining process if they receive a new valid block that attaches to the longest chain and they start to mine the following block;
the delay between a block discovery and its broadcast may allow other miners to complete and publish a conflicting block, effectively creating a fork in the blockchain.
In practice, an attacker can increase the amount of wasted blocks in the system and slow down the growth of the longest chain by simply delaying the propagation of new blocks \cite{balance_attack_2017}.

\bigskip
The Balance attack takes advantage of this idea.
The attacker partitions the Bitcoin nodes by disrupting communications between different subgroups:
the messages containing new blocks are delayed or possibly dropped completely.
The paper suggests to select groups of similar computing power to maximize the probability of success of the attack.
\cref{fig:balance} shows an example of Balance attack, where the attacker controls one router and is able to filter and manipulate traffic for the nodes connected to Internet through that router.
Multiple results recently demonstrated how simply a motivated attacker can perform networking attacks on the Internet and thus delaying messages in blockchain networks \cite{bgp_hijacking_secureworks, eclipse_attack_2015, hijacking_bitcoin_2017}.

\begin{figure}[h]
	\centering
	\vspace*{0.25cm}
	\includegraphics[scale=0.8]{figures/balance}
	\vspace*{0.25cm}
	\caption[Illustration of a network partitioning caused by a Balance Attack]{
		Illustration of a network partitioning caused by a Balance Attack.
		The attacker controls the bottom-left router (in red) and delays all communications between pair of nodes $(a, b)$ where $a \in \{7,8,9\}$ and $b \not \in \{7, 8, 9\}$:
		nodes inside the same partition can communicate with each other without any delay.
	}
	\label{fig:balance}
\end{figure}

In details, the attack works as follows:
\begin{itemize}
	\item the attacker identifies subgroups of miners with a similar mining power; different works in the literature \cite{deanonymization_2014, discovering_influential_nodes_2014} describe techniques to learn the network topology and identify influential nodes;
	\item the attacker identifies the routes of Bitcoin messages between different subgroups over Internet; let us assume the miner finds a partitions of nodes into \num{2} groups \texttt{G1} and \texttt{G2};
	\item the attacker runs some networking attack (man-in-the-middle, BGP hijacking \cite{bgp_hijacking_secureworks}) to take control of the communications links between groups and delays the Bitcoin traffic;
	\item because of the delay, transactions and blocks are not immediately propagated to the entire network, so \texttt{G1} and \texttt{G2} are very likely to have different views of the blockchain and to mine conflicting blocks;
	\item the attacker takes advantage of these inconsistencies and performs double-spending attacks.
\end{itemize}

The authors implemented and tested the attack against a private Ethereum test network with characteristics similar to \texttt{R3}, a consortium of more than \num{70} word-wide financial institutions.
The testnet consisted of \num{18} machines running Ethereum and mining blocks (the mining process in Ethereum is based on \ac{PoW} and is similar to the one of Bitcoin, at least for the matter of the Balance attack).
The authors report that a single machine needs to delay messages for \num{20} minutes to double spend, while an attacker controlling a third of the mining power only needs a delay of \num{4} minutes to achieve \num{94}\% of attack success rate.
