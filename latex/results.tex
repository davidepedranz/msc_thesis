% settings for the plots in this chapter
\def \myextraleftmargin {-0.2cm}
\def \myplotswitdth {0.95}

\chapter{Results}
\label{chapter:results}
Our simulator allows to simulate the Bitcoin protocol under different conditions:
it is possible to choose the size of the network and the delay of messages between nodes.
It is also easy to simulate network attacks by simply changing the transport layer used or by adding a new layer on top of the default one.

\medskip
This chapter covers our experiments with Bitcoin.
First, we analyze the protocol at rest to obtain a baseline of its performances;
then, we introduce delays on the communications between nodes to check if a degraded network infrastructure affects the protocol;
finally we simulate different variants of the Balance attack (\cref{sec:balance}) and compare their effects on Bitcoin.


\section{Settings}
We run each simulation multiple times with different seeds for the random number generators to obtain higher statistical significance for the results.
Each run simulates \num{3} hours of activity of the Bitcoin protocol.
The parameter that regulates the blocks creation rate of the miners is tuned so that a new block is generated on average every \num{10} minutes.
Since we do not simulate attacks that completely isolates nodes from each other or interfere with the control messages, we have disabled the ping-pong mechanism used by Bitcoin to verify that TCP connections are still alive:
this allowed us to spare computational resources and run the same simulation multiple times with different seeds.


\section{Parameters}
The simulator accepts different parameters that allows to change the settings of the simulation.
All parameters can be changed independently of each other and are specified in the configuration file.
In particular, the simulator accepts the following \num{5} values:
\begin{enumerate}
	\item \texttt{network\_size};
	\item \texttt{delay};
	\item \texttt{balance\_attack\_delay};
	\item \texttt{balance\_attack\_drop};
	\item \texttt{balance\_attack\_partitions}.
\end{enumerate}

\paragraph{Network Size}
The \texttt{network\_size} parameter controls the number of nodes simulated.
It can vary from \num{1} to a potentially unlimited number, which is only bounded by the available computational resources.
Since the Bitcoin network has about \num{9000} active node on average at the time of writing, we run simulations up to \num{10000} nodes.

\paragraph{Delay}
The \texttt{delay} parameter controls the delay of messages exchanged between Bitcoin nodes.
The delay of each message is randomly chooses between \num{1} and \num{5} times the specified value.

\paragraph{Balance Attack Delay}
The \texttt{balance\_attack\_delay} parameter controls the magnitude of the simulated Balance attack:
it adds an additional delay to all messages of type \texttt{Block} exchanged by nodes belonging to different partitions.
The delay is added to the base network delay controlled by the \texttt{delay} parameter.

\paragraph{Balance Attack Drop}
The \texttt{balance\_attack\_drop} parameter controls the drop rate of \texttt{Block} messages exchanged by nodes belonging to different partitions.
Please note that this is an extension to the Balance attack described in \cite{balance_attack_2017}, since the original paper does not consider dropping messages between nodes.

\paragraph{Balance Attack Partitions}
The \texttt{balance\_attack\_partitions} parameter controls the number of partitions of equal size in which the nodes of the network are divided by the Balance attack.
The setting discussed in the original paper is to partition the network in \num{2} parts.


\section{Evaluation}
Most attacks against Bitcoin aim at spending the double spending the money.
Forks on the blockchain allows an attacker to easily achieve double spending attack.
The main metric used to evaluate the performances of the protocol is thus the number of forks in the blockchain.
Forks happen in Bitcoin with a non-zero probability even at rest due to some inevitable delays of the information propagation;
degraded network conditions and a wide range of attacks can increase the probability of forks and open the possibility for easy double spend attempts.


\section{Experiments}

\subsection{Protocol at rest}
We run the simulations for different network sizes and $\texttt{delay} = 50 ~ms$, similar to the delay of a normal TCP connection over the Internet.
Under these conditions, the Bitcoin nodes exchange $8$ \texttt{Version}, \texttt{VerAck}, \texttt{GetAddr} and \texttt{Addr} messages each:
each node connects to exactly \num{8} peers and each connection generates an exchanges of the control messages.
\cref{fig:blocks-at-rest} shows that the network generated on average \num{18} blocks during the simulation of \num{3} hours (\num{1} block every \num{10} minutes), independently on the size of the network.
Under these conditions, the Bitcoin protocol does not produce any fork.

\begin{figure}[ht]
	\centering
	\advance \leftskip \myextraleftmargin
	\includegraphics[width=\myplotswitdth \columnwidth]{plots/blocks_at_rest}
	\caption[Blocks generation for networks of different sizes]{
		Blocks generation for networks of different sizes under normal conditions:
		\num{50} ms delay on average and no attack in progress.
		At time \num{0}, only the genesis block in present.
		During the simulation, a new block is generated on average every \num{10} minutes, regardless of the network's size.
	}
	\label{fig:blocks-at-rest}
\end{figure}

\subsection{Generalized network delays}
We run experiments for different network sizes and incremental delays from \num{0} to \num{30} seconds in the propagation of each message.
Generalized network delays on the network does not influence the number of control messages exchanged.
Similar to the previous experiment, the network generates \num{1} block every \num{10} minutes on average.
\cref{fig:forks-with-delay} and \cref{fig:forks-distribution-with-delay} show the effect of delays on the forks generated by a network of \num{9000} nodes:
with a \texttt{delay} of \num{0} seconds, the network never generates forks;
with small delays, the network generates forks only in a small number of simulations;
with longer delays, the network generates on average about \num{1} fork every \num{3} hours.

\begin{figure}[ht]
	\centering
	\advance \leftskip \myextraleftmargin
	\includegraphics[width=\myplotswitdth \columnwidth]{plots/forks_with_delay_9000}
	\caption[Average forks number for a network of 9000 nodes with different delays]{
		Average forks number for a network of 9000 nodes with different delays.
		In perfect conditions of zero delay, Bitcoin nodes never generate forks.
		With short delays, the average number of forks is small.
		With longer delays, Bitcoin generates on average about \num{1} every \num{3} hours.
	}
	\label{fig:forks-with-delay}
\end{figure}

\begin{figure}[ht]
	\centering
	\advance \leftskip \myextraleftmargin
	\includegraphics[width=\myplotswitdth \columnwidth]{plots/forks_with_delay_boxplot_9000}
	\caption[Forks distribution for a network of 9000 nodes with different delays]{
		Forks distribution for a network of 9000 nodes with different delays.
		With zero delay, blocks are immediately distributed to all nodes and the network does not generate any fork.
		Even a small delay of \num{5} s starts to create some forks.
		Higher delays produce an higher number of forks.
	}
	\label{fig:forks-distribution-with-delay}
\end{figure}

Similar results yield for smaller networks:
\cref{fig:forks-with-delay-small-network} and \cref{fig:forks-distribution-with-delay-small-network} show the forks number over time and their distribution at the end of the simulation for a network of \num{1000} nodes.

\begin{figure}[ht]
	\centering
	\advance \leftskip \myextraleftmargin
	\includegraphics[width=\myplotswitdth \columnwidth]{plots/forks_with_delay_1000}
	\caption[Average forks number for a network of 1000 nodes with different delays]{
		Average forks number for a network of \num{1000} nodes with different delays.
		In perfect conditions of zero delay, Bitcoin nodes never generate forks.
		With short delays, the average number of forks is small.
		With longer delays, Bitcoin generates on average about \num{1} every \num{3} hours.
	}
	\label{fig:forks-with-delay-small-network}
\end{figure}

\begin{figure}[ht]
	\centering
	\advance \leftskip \myextraleftmargin
	\includegraphics[width=\myplotswitdth \columnwidth]{plots/forks_with_delay_boxplot_1000}
	\caption[Forks distribution for a network of 1000 nodes with different delays]{
		Forks distribution for a network of \num{1000} nodes with different delays.
		With zero delay, blocks are immediately distributed to all nodes and the network does not generate any fork.
		Even a small delay of \num{5} s starts to create some forks.
		Higher delays produce an higher number of forks.
	}
	\label{fig:forks-distribution-with-delay-small-network}
\end{figure}


\subsection{Balance attack}

