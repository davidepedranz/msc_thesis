\chapter{Experiments \& Results}
\label{chapter:results}
Our simulator allows to simulate the Bitcoin protocol under different conditions:
it is possible to choose the size of the network and the delay of messages between nodes.
It is also easy to simulate network attacks by simply changing the transport layer used or by adding a new layer on top of the default one.

\medskip
This chapter covers our experiments with Bitcoin.
First, we analyze the protocol at rest to obtain a baseline of its performances;
then, we introduce delays on the communications between nodes to check if a degraded network infrastructure affects the protocol;
finally we simulate different variants of the Balance attack (\cref{sec:balance}) and compare their effects on Bitcoin.


\section{Settings}
We run each simulation multiple times with different seeds for the random number generators to obtain higher statistical significance for the results.
Each run simulates \num{3} hours of activity of the Bitcoin protocol.
The parameter that regulates the blocks creation rate of the miners is tuned so that a new block is generated on average every \num{10} minutes.
Since we do not simulate attacks that completely isolates nodes from each other or interfere with the control messages, we have disabled the ping-pong mechanism used by Bitcoin to verify that TCP connections are still alive:
this allowed us to spare computational resources and run the same simulation multiple times with different seeds.


\section{Parameters}
The simulator accepts different parameters that allows to change the settings of the simulation.
All parameters can be changed independently of each other and are specified in the configuration file.
In particular, the simulator accepts the following \num{5} values:
\begin{enumerate}
	\item \texttt{network\_size};
	\item \texttt{delay};
	\item \texttt{balance\_attack\_delay};
	\item \texttt{balance\_attack\_drop};
	\item \texttt{balance\_attack\_partitions}.
\end{enumerate}

\paragraph{Network Size}
The \texttt{network\_size} parameter controls the number of nodes simulated.
It can vary from \num{1} to a potentially unlimited number, which is only bounded by the available computational resources.
Since the Bitcoin network has about \num{9000} active node on average at the time of writing, we run simulations up to \num{10000} nodes.

\paragraph{Delay}
The \texttt{delay} parameter controls the delay of messages exchanged between Bitcoin nodes.
The delay of each message is randomly chooses between \num{1} and \num{5} times the specified value.

\paragraph{Balance Attack Delay}
The \texttt{balance\_attack\_delay} parameter controls the magnitude of the simulated Balance attack:
it adds an additional delay to all messages of type \texttt{Block} exchanged by nodes belonging to different partitions.
The delay is added to the base network delay controlled by the \texttt{delay} parameter.

\paragraph{Balance Attack Drop}
The \texttt{balance\_attack\_drop} parameter controls the drop rate of \texttt{Block} messages exchanged by nodes belonging to different partitions.
Please note that this is an extension to the Balance attack described in \cite{balance_attack_2017}, since the original paper does not consider dropping messages between nodes.

\paragraph{Balance Attack Partitions}
The \texttt{balance\_attack\_partitions} parameter controls the number of partitions of equal size in which the nodes of the network are divided by the Balance attack.
The setting discussed in the original paper is to partition the network in \num{2} parts.


\section{Protocol at rest}
We start with a
